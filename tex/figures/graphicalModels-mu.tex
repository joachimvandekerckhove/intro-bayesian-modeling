% An example graphical model, based on a general(ish) set of definitions
% Michael Lee (mdlee@uci.edu)
% Jan 2018
% 
% to convert to eps download GNU ps tools (https://sourceforge.net/projects/gnuwin32/?source=typ_redirect)
% then use (MSDOS) command line of form
% pdftops -eps graphicalModel.pdf test.eps
% to do second page, for example ...
% pdftops -eps -f 2 -l 2 graphicalModel.pdf test.eps

\documentclass[12pt]{article}

% packages
\usepackage{tikz}
\usepackage{array}
\usepackage{amsmath}
\usepackage{bm}
\usetikzlibrary{shadows,fit,positioning,arrows,intersections}

\usepackage{tikz-qtree}
\usetikzlibrary{matrix,chains,positioning,decorations.pathreplacing,
                arrows,positioning,trees,fit}
\usetikzlibrary{trees}
% Set the overall layout of the tree
\tikzstyle{level 1}=[level distance=2.5cm, sibling distance=3.5cm]
\tikzstyle{level 2}=[level distance=2.5cm, sibling distance=2cm]

% Define styles for bags and leafs
\tikzstyle{bag} = [text width=4em, text centered]
\tikzstyle{end} = [circle, minimum width=3pt,fill, inner sep=0pt]

% stuff from somewhere
\makeatletter
\pgfkeys{/pgf/.cd,
  rectangle corner radius/.initial=3pt
}
\newif\ifpgf@rectanglewrc@donecorner@
\def\pgf@rectanglewithroundedcorners@docorner#1#2#3#4{%
  \edef\pgf@marshal{%
    \noexpand\pgfintersectionofpaths
      {%
        \noexpand\pgfpathmoveto{\noexpand\pgfpoint{\the\pgf@xa}{\the\pgf@ya}}%
        \noexpand\pgfpathlineto{\noexpand\pgfpoint{\the\pgf@x}{\the\pgf@y}}%
      }%
      {%
        \noexpand\pgfpathmoveto{\noexpand\pgfpointadd
          {\noexpand\pgfpoint{\the\pgf@xc}{\the\pgf@yc}}%
          {\noexpand\pgfpoint{#1}{#2}}}%
        \noexpand\pgfpatharc{#3}{#4}{\cornerradius}%
      }%
    }%
  \pgf@process{\pgf@marshal\pgfpointintersectionsolution{1}}%
  \pgf@process{\pgftransforminvert\pgfpointtransformed{}}%
  \pgf@rectanglewrc@donecorner@true
}
\pgfdeclareshape{rectangle with rounded corners}
{
  \inheritsavedanchors[from=rectangle] % this is nearly a rectangle
  \inheritanchor[from=rectangle]{north}
  \inheritanchor[from=rectangle]{north west}
  \inheritanchor[from=rectangle]{north east}
  \inheritanchor[from=rectangle]{center}
  \inheritanchor[from=rectangle]{west}
  \inheritanchor[from=rectangle]{east}
  \inheritanchor[from=rectangle]{mid}
  \inheritanchor[from=rectangle]{mid west}
  \inheritanchor[from=rectangle]{mid east}
  \inheritanchor[from=rectangle]{base}
  \inheritanchor[from=rectangle]{base west}
  \inheritanchor[from=rectangle]{base east}
  \inheritanchor[from=rectangle]{south}
  \inheritanchor[from=rectangle]{south west}
  \inheritanchor[from=rectangle]{south east}

  \savedmacro\cornerradius{%
    \edef\cornerradius{\pgfkeysvalueof{/pgf/rectangle corner radius}}%
  }

  \backgroundpath{%
    \northeast\advance\pgf@y-\cornerradius\relax
    \pgfpathmoveto{}%
    \pgfpatharc{0}{90}{\cornerradius}%
    \northeast\pgf@ya=\pgf@y\southwest\advance\pgf@x\cornerradius\relax\pgf@y=\pgf@ya
    \pgfpathlineto{}%
    \pgfpatharc{90}{180}{\cornerradius}%
    \southwest\advance\pgf@y\cornerradius\relax
    \pgfpathlineto{}%
    \pgfpatharc{180}{270}{\cornerradius}%
    \northeast\pgf@xa=\pgf@x\advance\pgf@xa-\cornerradius\southwest\pgf@x=\pgf@xa
    \pgfpathlineto{}%
    \pgfpatharc{270}{360}{\cornerradius}%
    \northeast\advance\pgf@y-\cornerradius\relax
    \pgfpathlineto{}%
  }

  \anchor{before north east}{\northeast\advance\pgf@y-\cornerradius}
  \anchor{after north east}{\northeast\advance\pgf@x-\cornerradius}
  \anchor{before north west}{\southwest\pgf@xa=\pgf@x\advance\pgf@xa\cornerradius
    \northeast\pgf@x=\pgf@xa}
  \anchor{after north west}{\northeast\pgf@ya=\pgf@y\advance\pgf@ya-\cornerradius
    \southwest\pgf@y=\pgf@ya}
  \anchor{before south west}{\southwest\advance\pgf@y\cornerradius}
  \anchor{after south west}{\southwest\advance\pgf@x\cornerradius}
  \anchor{before south east}{\northeast\pgf@xa=\pgf@x\advance\pgf@xa-\cornerradius
    \southwest\pgf@x=\pgf@xa}
  \anchor{after south east}{\southwest\pgf@ya=\pgf@y\advance\pgf@ya\cornerradius
    \northeast\pgf@y=\pgf@ya}

  \anchorborder{%
    \pgf@xb=\pgf@x% xb/yb is target
    \pgf@yb=\pgf@y%
    \southwest%
    \pgf@xa=\pgf@x% xa/ya is se
    \pgf@ya=\pgf@y%
    \northeast%
    \advance\pgf@x by-\pgf@xa%
    \advance\pgf@y by-\pgf@ya%
    \pgf@xc=.5\pgf@x% x/y is half width/height
    \pgf@yc=.5\pgf@y%
    \advance\pgf@xa by\pgf@xc% xa/ya becomes center
    \advance\pgf@ya by\pgf@yc%
    \edef\pgf@marshal{%
      \noexpand\pgfpointborderrectangle
      {\noexpand\pgfqpoint{\the\pgf@xb}{\the\pgf@yb}}
      {\noexpand\pgfqpoint{\the\pgf@xc}{\the\pgf@yc}}%
    }%
    \pgf@process{\pgf@marshal}%
    \advance\pgf@x by\pgf@xa% 
    \advance\pgf@y by\pgf@ya%
    \pgfextract@process\borderpoint{}%
    %
    \pgf@rectanglewrc@donecorner@false
    %
    % do southwest corner
    \southwest\pgf@xc=\pgf@x\pgf@yc=\pgf@y
    \advance\pgf@xc\cornerradius\relax\advance\pgf@yc\cornerradius\relax 
    \borderpoint
    \ifdim\pgf@x<\pgf@xc\relax\ifdim\pgf@y<\pgf@yc\relax
      \pgf@rectanglewithroundedcorners@docorner{-\cornerradius}{0pt}{180}{270}%
    \fi\fi
    %
    % do southeast corner
    \ifpgf@rectanglewrc@donecorner@\else
      \southwest\pgf@yc=\pgf@y\relax\northeast\pgf@xc=\pgf@x\relax
      \advance\pgf@xc-\cornerradius\relax\advance\pgf@yc\cornerradius\relax
      \borderpoint
      \ifdim\pgf@x>\pgf@xc\relax\ifdim\pgf@y<\pgf@yc\relax
       \pgf@rectanglewithroundedcorners@docorner{0pt}{-\cornerradius}{270}{360}%
      \fi\fi
    \fi
    %
    % do northeast corner
    \ifpgf@rectanglewrc@donecorner@\else
      \northeast\pgf@xc=\pgf@x\relax\pgf@yc=\pgf@y\relax
      \advance\pgf@xc-\cornerradius\relax\advance\pgf@yc-\cornerradius\relax
      \borderpoint
      \ifdim\pgf@x>\pgf@xc\relax\ifdim\pgf@y>\pgf@yc\relax
       \pgf@rectanglewithroundedcorners@docorner{\cornerradius}{0pt}{0}{90}%
      \fi\fi
    \fi
    %
    % do northwest corner
    \ifpgf@rectanglewrc@donecorner@\else
      \northeast\pgf@yc=\pgf@y\relax\southwest\pgf@xc=\pgf@x\relax
      \advance\pgf@xc\cornerradius\relax\advance\pgf@yc-\cornerradius\relax
      \borderpoint
      \ifdim\pgf@x<\pgf@xc\relax\ifdim\pgf@y>\pgf@yc\relax
       \pgf@rectanglewithroundedcorners@docorner{0pt}{\cornerradius}{90}{180}%
      \fi\fi
    \fi
  }
}

% Document starts here
\begin{document}

% use the whole page (especially for this example model, which needs it)
\setlength{\parindent}{-4cm}

% Colors for observed and partially observed nodes
\definecolor{lightgray}{RGB}{180,180,180}
\definecolor{verylightgray}{RGB}{220,220,220}

% Definitions of node types
\tikzstyle{discreteDeterministic} = [draw,  minimum size = 13mm, thick, draw = black, node distance = 10mm, fill  = white, line width = 1pt]
\tikzstyle{discreteDeterministicObserved} = [draw,  minimum size = 13mm, thick, draw = black, node distance = 10mm, fill  = white, line width = 1pt]
\tikzstyle{discreteObserved} = [draw,  minimum size = 11mm, thick, draw = black, node distance = 10mm, fill  = lightgray, line width = 1pt]
\tikzstyle{discretePartialObserved} = [draw,  minimum size = 11mm, thick, draw = black, node distance = 10mm, fill  = verylightgray, line width = 1pt]
\tikzstyle{discreteLatent} = [draw,  minimum size = 11mm, thick, draw = black, node distance = 10mm, fill  = white, line width = 1pt]
\tikzstyle{continuousDeterministic} = [circle,  minimum size = 14mm, thick, draw = black, node distance = 10mm, fill  = white, line width = 1pt]
\tikzstyle{continuousDeterministicObserved} = [circle,  minimum size = 14mm, thick, draw = black, node distance = 10mm, fill  = white, line width = 1pt]
\tikzstyle{continuousObserved} = [circle, minimum size = 12mm, thick, draw = black, node distance = 10mm, fill = lightgray, line width = 1pt]
\tikzstyle{continuousPartialObserved} = [circle, minimum size = 12mm, thick, draw = black, node distance = 10mm, fill = verylightgray, line width = 1pt]
\tikzstyle{continuousLatent} = [circle, minimum size = 12mm, thick, draw = black, node distance = 10mm, line width = 1pt]
\tikzstyle{plate} = [rectangle with rounded corners, draw = black, rectangle corner radius = 6pt, align = center, line width = 1pt, yshift = -4]
\tikzstyle{spacer} = [opacity = 0, minimum size = 1mm] {};
\tikzstyle{connect} = [-stealth, thick, line width = 1pt]

\pagestyle{empty}

  %----------------------------------------------------------------------------------
  % Single high threshold, conceptual
  %----------------------------------------------------------------------------------
  
\begin{tabular}{ m{8.5cm} m{5.5cm}}
  \begin{tikzpicture}
\tikzset{grow'=right}
\tikzset{execute at begin node=\strut}
\tikzset{every tree node/.style={anchor=base west}}
\tikzset{level 1/.style={level distance=60pt}}
\tikzset{level 2/.style={level distance=60pt}}
\tikzset{level 3+/.style={level distance=60pt}}
\Tree [.``old'' 	[.$\rho$ ``hit'' ]
              		[.$\left(1-\rho\right)$ 	[.$\gamma$ ``hit''  ]
						[.$\left(1-\gamma\right)$ ``miss'' ]
 ] ] 
\end{tikzpicture}
&

 \begin{tikzpicture}
\tikzset{grow'=right}
\tikzset{execute at begin node=\strut}
\tikzset{every tree node/.style={anchor=base west}}
\tikzset{level 1/.style={level distance=60pt}}
\tikzset{level 2/.style={level distance=60pt}}
\tikzset{level 3+/.style={level distance=60pt}}
\Tree [.``new''		 [.$\gamma$ {``false alarm''} ]
              			[.$\left(1-\gamma\right)$ {``correct rejection''} ]
 ] ] 
\end{tikzpicture}

\end{tabular}



 %----------------------------------------------------------------------------------
  % Two high threshold, conceptual
  %----------------------------------------------------------------------------------
  
  \newpage
\begin{tabular}{ m{8.5cm} m{5.5cm}}
  \begin{tikzpicture}
\tikzset{grow'=right}
\tikzset{execute at begin node=\strut}
\tikzset{every tree node/.style={anchor=base west}}
\tikzset{level 1/.style={level distance=60pt}}
\tikzset{level 2/.style={level distance=60pt}}
\tikzset{level 3+/.style={level distance=60pt}}
\Tree [.``old'' 	[.$\rho$ ``hit'' ]
              		[.$\left(1-\rho\right)$ 	[.$\gamma$ ``hit''  ]
						[.$\left(1-\gamma\right)$ ``miss'' ]
 ] ] 
\end{tikzpicture}
&

 \begin{tikzpicture}
\tikzset{grow'=right}
\tikzset{execute at begin node=\strut}
\tikzset{every tree node/.style={anchor=base west}}
\tikzset{level 1/.style={level distance=60pt}}
\tikzset{level 2/.style={level distance=60pt}}
\tikzset{level 3+/.style={level distance=60pt}}
\Tree [.``new'' 	[.$\rho$ {``correct rejection'}' ]
              		[.$\left(1-\rho\right)$ 	[.$\gamma$ {``false alarm''}  ]
						[.$\left(1-\gamma\right)$ {``correct rejection''} ]
 ] ] 
\end{tikzpicture}

\end{tabular}



 %----------------------------------------------------------------------------------
  % Weapon priming, conceptual, based heavily on Heck et al Figure 3
  %----------------------------------------------------------------------------------
  
  \newpage

 \begin{tikzpicture}[grow=right]
    \tikzset{grow'=right, level distance=55pt,
    		% level 1/.style ={level distance=3cm}
    		 %level 4/.style ={level distance=1.5cm}
    		 }
    		   \tikzstyle{level 1}=[sibling distance=5mm, level distance=3cm]
    		     \tikzstyle{level 2}=[sibling distance=5mm, level distance=2cm]
    		   \tikzstyle{level 3}=[sibling distance=5mm, level distance=2cm]
    \tikzset{execute at begin node=\strut}
    \tikzset{every tree node/.style={anchor=base west}}
    \Tree [.\node[](Stroop){\textbf{Stimulus}}; 
		    [.\node[draw=none](A){$\alpha$};
		    	[.\node{};
		    		[ .\node{};
		    			[.\node(c1)[]{}; ]
		    		]
		    	]
		    ]
		    [.$1-\alpha$
				[.\node[draw=none](C){$\beta$};
					[ .\node{};
						[.\node(c2)[]{}; ]
					]   
				]
				[.$1-\beta$ 
					[.$\gamma$   {}   ]
					[.$1-\gamma$ {}  ] 
				]
			] 
		  ]
\draw[-] (A)--(c1);
\draw[-] (C)--(c2);
\newcommand\sibdist{2.5cm}
\matrix(m)[% table as tikz matrix
        matrix of nodes,
        xshift=11cm,
        yshift=0.575cm,
        nodes={inner xsep=\tabcolsep,minimum height=0.5*\sibdist},
      ]{
        \textbf{Prime:}  & White & White & Black & Black\\
        \textbf{Target:} & Tool  & Gun   & Tool  & Gun\\
        & $+$ & $-$ & $-$ & $+$ \\ %[.5*\sibdist]
        & $+$ & $+$ & $+$ & $+$ \\
        & $+$ & $-$ & $+$ & $-$ \\
        & $-$ & $+$ & $-$ & $+$ \\
    };
    \foreach \i in {2,...,5}\node[draw,fit=(m-1-\i) (m-2-\i) (m-3-\i) (m-4-\i)(m-5-\i)(m-6-\i),inner sep=-.5\pgflinewidth](c\i){};
\end{tikzpicture}



  
\end{document}

